\documentclass[12pt,a4paper]{article}

% USEPACKAGE LISTA
\usepackage[utf8]{inputenc}
\usepackage{amsmath}
\usepackage{mathtools}
\usepackage{marvosym} 
\usepackage{wrapfig}
\usepackage[usenames,dvipsnames,table]{xcolor}
\usepackage{hyperref}
\usepackage{float}
\usepackage{multicol}
\hypersetup{colorlinks,citecolor=black,filecolor=black,linkcolor=blue,urlcolor=blue}
\usepackage{pdfpages}
\usepackage{amsfonts}
\usepackage{amssymb}
\usepackage{fancyhdr}
\usepackage{graphicx}
\usepackage{t1enc}
\usepackage[magyar]{babel}
\usepackage{bm}

\usepackage{pgfplots}
\pgfplotsset{height = 10cm, width=15cm,compat=1.9}

\usepackage[left=2cm,right=2cm,top=2cm,bottom=2cm]{geometry}

\setlength{\parindent}{0pt}
\setlength{\parskip}{0em}
\setlength{\arrayrulewidth}{0.25mm}
\pagestyle{fancy}
\fancyhf{}

\title{Tantárgyi információk}
\author{A többváltozós analízis mérnöki alkalmazásai - X1 kurzus}
\date{2022/23/II. (tavaszi) félév}

\lhead{A többváltozós analízis mérnöki alkalmazásai}
\chead{}
\rhead{2022/23/II. félév}
\cfoot{\thepage. oldal}

% ITT KEZDŐDIK A DOKUMENTUM
\begin{document}
\maketitle{}
\thispagestyle{empty}
\section{A tantárgy célkitűzése}
A tantárgy célja az, hogy a hallgatókkal \textit{megismertesse a Python programozás alapjait}, miközben betekintést ad az ipari gyakorlatban előforduló problémákba, így például az \textit{adatkezelésbe}, a \textit{képfeldolgozásba} és többek között a \textit{mesterséges intelligencia alkalmazásába}. Emellett az is a tárgy célkitűzése, hogy ezeket elsősorban projektfeladatokon és gyakorlati alkalmazásokon keresztül sajátítsák el a hallgatók.
\section{Ütemterv}
\rowcolors{2}{blue!30!white!70}{white!50!blue!60}
\begin{table}[h!]
    \centering\vspace{-1em}
    \caption{A félév tematikája heti lebontásban}\vspace{0.5em}
    \begin{tabular}{|c|c|l|}
        \hline
        \#. \textbf{hét} & \textbf{Dátum} & \textbf{Témakör} \\\hline\hline
        1. hét $\dagger$  & 02. 27. & Bevezetés a programozásba; Numpy I. \\ \hline
        2. hét $\ddagger$ & 03. 06. & Függvények bevezetése; Grafikonok (Matplotlib); Numpy II. \\ \hline
        3. hét $\ddagger$ & 03. 13. & Feltételes (if) elágazások; Mechanikai példamegoldás (Sympy) \\ \hline
        4. hét $\dagger$ & 03. 20. & Ciklusok; Az objektumorientált programozás alapjai\\ \hline
        5. hét $\dagger$ & 03. 27. & Filekezelés; Adatbázisok (Pandas) \\ \hline
        6. hét $\ddagger$ & 04. 03. & Algoritmusok és folyamatok komplexitása \\ \hline
        7. hét & 04. 10. & - \\ \hline\hline
        8. hét $\ddagger$ & 04. 17. & Képfeldolgozás I. : színmodellek (RGB, BGR), képmanipuláció \\ \hline
        9. hét $\ddagger$ & 04. 24. & Képfeldolgozás II. : Objektumok detektálása (él-, alakfelismerés) \\ \hline
        10. hét & 05. 01. & ($\star$) Fakultatív Előadás 1: Elemi bevezetés az adattudományokba \\ \hline
        11. hét $\dagger$ & 05. 08. & AI és Deep Learning I. \\ \hline
        12. hét $\dagger$ & 05. 15. & AI és Deep Learning II. \\ \hline
        13. hét & 05. 22. & ($\star$) Fakultatív előadás 2: A programozás szerepe az iparban \\ \hline
        14. hét & 05. 29 & - \\ \hline
    \end{tabular}
\end{table}
\begin{scriptsize}
    \vspace{-2em}\hspace{1em}
    $\dagger$ - Wenesz Dominik $|$ $\ddagger$ - Monori Bence
\end{scriptsize}
\section{Teljesítményértékelések}
A tárgy teljesítéséhez \textbf{2 db}, egyenként 50-50 pontos \textbf{Házi feladat} leadása szükséges:
\begin{itemize}
    \item Az \textbf{1. HF} egy mechanikai probléma megoldásából, és a hozzátartozó grafikonok elkészítéséből áll. Ennek határideje a \textit{8. heti gyakorlat vége}: 04.17. - 17:59
    \item A \textbf{2. HF} egy szabadon választott projekt megvalósítása. Ennek határideje a \textit{14. heti gyakorlat vége}: 05.29. - 17:59
    \item A házi feladatnak tartalmaznia kell a hallgató(k) \textit{nevét} és \textit{Neptun-kódját}, emellett a kódnak körülbelül 15-20\% \textit{kommentet} kell tartalmaznia. (Ez a gyakorlatban azt jelenti, hogy minden 4-5 sorra jutnia kell valamilyen magyarázószövegnek.)
    \item A házi feladatok teljesítésére önállóan és \textit{2 fős párok}ban is van lehetőség. Fontos, hogy az utóbbi esetben a páros mindkét tagjának le kell adnia a házi feladatot a Teams rendszeren keresztül.
    \item A házi feladatokat a tantárgy hivatalos \textbf{Teams csatornáján} kell a megfelelő \textbf{assignment}nél leadni a megadott határidő végéig. Itt kizárólag a feladat kódját tartalmazó \textit{.ipynb} (\textit{i}nteractive \textit{py}thon \textit{n}ote\textit{b}ook) leadása szükséges. Egyéni feladat esetén a feladatkiírásnak is csatolva kell lennie.
    \item Pótleadási határidő mindkét házi feladat esetén \textit{póthét szerda (06. 07.) - 23:59}. Pótleadás esetén a pontok legfeljebb 80\%-át lehet megszerezni, viszont díjbefizetési kötelezettséggel ez nem jár!
\end{itemize}
A tárgy során emellett még lehetőség van \textbf{szorgalmi feladat}ok teljesítésére, amelyekkel további pluszpontokat lehet szerezni. Ennek elsődleges célja az, hogy a hallgatók könnyebben el tudják mélyíteni az órák anyagát.\\[10pt]
A fentiek figyelembevételével a félév végi érdemjegyeket az alábbiak szerint határozzuk meg:
\rowcolors{2}{blue!0!white!70}{white!50!blue!0}
\begin{table}[h]
    \centering
    \begin{tabular}{|c|c|}
        \hline
        85 - & 5 \\ \hline
        70 - 84 & 4 \\ \hline
        55 - 69 & 3 \\ \hline
        40 - 54 & 2 \\ \hline
        - 39 & 1 \\ \hline
    \end{tabular}
\end{table}
\section{Elérhetőségek}
A megnevezett félévben a tárgyat \textit{Monori Bence} és \textit{Wenesz Dominik} tartják. Az elérhetőségeik:
\begin{itemize}
    \item Monori Bence: \texttt{m.bence02@outlook.hu}
    \item Wenesz Dominik: \texttt{weneszdominik@gmail.com}
\end{itemize}
Emellett a félév során lehetőségetek van arra, hogy bármikor anonim módon kérdezzetek, vagy jelezzétek a problémátokat felénk, ezt az alábbi formon éritek el: \texttt{\url{https://forms.gle/Qvj7okQqCMRc4cBu7}}.
\end{document}